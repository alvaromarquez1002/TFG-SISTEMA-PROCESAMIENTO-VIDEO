\capitulo{2}{Objetivos del proyecto}

En este capítulo se definen las metas y los objetivos que se han perseguido a lo largo del desarrollo de este Trabajo de Fin de Grado. Estos se han dividido en objetivos generales, que describen el propósito principal del proyecto, y objetivos técnicos, que detallan las metas específicas a nivel de implementación y tecnología. Finalmente, se incluye una sección con los objetivos personales que se han originado con la realización de este trabajo.

\section{Objetivos generales}

El objetivo principal de este proyecto es el diseño y la implementación de una infraestructura de \textit{backend}para un sistema de procesamiento de vídeo destinado a la tele-rehabilitación de pacientes con la enfermedad de Parkinson.

Este objetivo general se fundamenta en la necesidad de evolucionar un sistema preexistente, asegurando que la nueva arquitectura pueda manejar de forma fiable la captura, el transporte y el almacenamiento de los flujos de datos de vídeo, sentando así las bases para futuras implementaciones de algoritmos de análisis y \textit{feedback} automático.

\section{Objetivos técnicos}

Para alcanzar el objetivo general, se establecieron las siguientes metas técnicas específicas:

\begin{itemize}
    \item Investigar y seleccionar un conjunto de tecnologías de código abierto adecuadas para construir un \textit{pipeline} de datos en \textit{streaming}, centrándose en sus licencias.
    
    \item Desplegar y configurar un sistema de captura de vídeo basado en \textbf{Jitsi}, utilizando su componente \textbf{Jibri} para la grabación de las sesiones de ejercicios en formato de archivo (MP4).
    
    \item Utilizar \textbf{Apache Kafka} como un bus de mensajería distribuido para desacoplar los componentes del sistema y garantizar un transporte de datos fiable.
    
    \item Integrar \textbf{Apache Spark} en la arquitectura como motor de procesamiento, preparando el sistema para el futuro análisis de los datos de vídeo almacenados o en \textit{streaming}.
    
    \item Contenerizar toda la infraestructura utilizando \textbf{Docker} y \textbf{Docker Compose}, con el fin de crear un entorno de despliegue consistente, reproducible y fácil de gestionar.
    
    \item Documentar de manera exhaustiva todo el proceso de investigación, diseño, implementación y los desafíos técnicos encontrados, utilizando \LaTeX{} a través de la plataforma Overleaf.
    
    \item Utilizar herramientas modernas de gestión de proyectos (\textbf{Jira}) y control de versiones (\textbf{Git/GitHub}) para llevar un seguimiento riguroso del trabajo realizado y asegurar la integridad del código fuente y la documentación.
    
\end{itemize}


\section{Objetivos personales}

Más allá de los requisitos técnicos, la realización de este TFG permite perseguir una serie de objetivos personales y de desarrollo profesional:

\begin{itemize}
    \item Aplicar los conocimientos teóricos adquiridos a un problema de ingeniería complejo y real.
    \item Profundizar y ganar experiencia práctica en tecnologías de alta demanda en la industria, como son las plataformas(Kafka, Spark) y las herramientas (Docker).
    \item Afrontar y resolver problemas técnicos, desarrollando la capacidad de análisis, depuración y toma de decisiones ante imprevistos, como los surgidos durante el despliegue de la infraestructura.
    \item Contribuir, aunque sea a nivel de infraestructura, a un proyecto con un impacto social positivo y relevante.
\end{itemize}