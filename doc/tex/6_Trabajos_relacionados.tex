\chapter{Trabajos relacionados}
\label{chap:trabajos_relacionados}

Para contextualizar adecuadamente el alcance y la contribución de este TFG, es fundamental analizar los proyectos e investigaciones previas que han servido como base o que abordan problemas similares. Este capítulo revisa los trabajos más influyentes, desde el proyecto directo del que este TFG es una continuación, hasta investigaciones académicas que validan el uso de las tecnologías seleccionadas.

\section{Proyecto de Origen: El Sistema FIS-FBIS}
\label{sec:trab_rel_fisfbis}
Este Trabajo de Fin de Grado es una continuación directa del proyecto \textbf{FIS-FBIS} \cite{garrido_fisfbis}, un sistema de bajo coste para la telerehabilitación de pacientes con Parkinson desarrollado en la Universidad de Burgos. El objetivo de FIS-FBIS era crear una plataforma completa que permitiera la comunicación entre paciente y terapeuta, la grabación de sesiones y, fundamentalmente, la evaluación automática de los ejercicios.

El sistema original fue el resultado de dos Trabajos de Fin de Máster que funcionaban de manera conjunta y complementaria: uno enfocado en la infraestructura de datos (TFM-FIS-IF) y otro en la inteligencia artificial para el análisis de vídeo (TFM-FIS-IA). Aquí proporciono el articulo científico relacionado \cite{garrido2023fishub}.

\subsection{TFM-FIS-IF: Infraestructura de Datos para Procesamiento de Vídeo}
El trabajo de José Luis Garrido Labrador \cite{garrido_tfm_if} se centró en la creación de una arquitectura Big Data basada en colas de mensajería para el procesado de vídeo en tiempo real. Su sistema ya utilizaba \textbf{Apache Kafka} como bus de eventos y \textbf{Apache Spark} para el procesamiento. La ingesta de datos se realizaba mediante scripts de Python (`emitter.py`, `producer.py`) que simulaban un flujo de vídeo y lo enviaban a Kafka para ser procesado por un consumidor de Spark (`consumer.py`).

Si bien este TFM sentó las bases de la arquitectura de este proyecto, se trataba de una prueba de concepto.Este TFG toma esta arquitectura como punto de partida y la evoluciona, reemplazando los scripts de simulación por una integración real con \textbf{Jitsi/Jibri} y profesionalizando todo el despliegue mediante \textbf{Docker y Docker Compose} para garantizar un entorno robusto, escalable y fácil de gestionar.

\subsection{TFM-FIS-IA: Visión Artificial para Comparación de Posiciones}
De forma paralela, el trabajo de José Miguel Ramírez Sanz \cite{ramirez_tfm_ia} abordó la parte de inteligencia artificial del sistema. Su TFM se centró en el estudio del estado del arte de las herramientas de visión artificial para Python y en la implementación de un sistema de comparación de posiciones. El objetivo era extraer el esqueleto humano de los fotogramas de vídeo para poder comparar los ejercicios realizados por los pacientes con los de los terapeutas.

La sinergia entre ambos TFMs es clara: el TFM de infraestructura (IF) proporcionaba el pipeline de datos, mientras que el de inteligencia artificial (IA) aportaba la lógica de análisis. Este trabajo se centra en mejorar y modernizar la parte de infraestructura (IF) para que pueda dar un soporte más sólido a futuras aplicaciones de análisis como la desarrollada en el TFM-IA.

\subsection{Arquitecturas para el Análisis de Vídeo en Tiempo Real con Spark}
\label{subsec:trab_rel_academico}
Más allá de los proyectos del entorno de la UBU, se ha revisado la literatura académica para validar la elección de la arquitectura. Un ejemplo es el artículo \textit{Distributed Real-Time Video Stream Analytics on top of Spar}k \cite{karimov2018distributed}. En este trabajo, los autores proponen una arquitectura genérica para el análisis de flujos de vídeo en tiempo real utilizando Spark Streaming.

El estudio demuestra cómo Spark es capaz de procesar flujos de datos para realizar tareas como la detección de objetos. Su enfoque valida la elección de Spark como motor de procesamiento para este TFG, dada su capacidad para manejar la carga computacional del análisis de vídeo de forma distribuida. Sin embargo, este proyecto se diferencia y amplía este enfoque al:
\begin{itemize}
    \item Integrar una solución completa de captura de vídeo para telemedicina como es \textbf{Jitsi/Jibri}, en lugar de fuentes de vídeo genéricas.
    \item Implementar \textbf{Apache Kafka} como un bus de mensajería intermedio, lo que proporciona un mayor desacoplamiento y tolerancia a fallos entre la captura y el procesamiento.
    \item Enfocar la arquitectura en un caso de uso específico (la telerehabilitación), en lugar de en el análisis genérico de vídeo.
\end{itemize}