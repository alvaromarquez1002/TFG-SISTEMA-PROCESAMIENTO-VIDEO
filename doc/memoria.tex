\documentclass[a4paper,12pt,twoside]{memoir}

% Castellano
\usepackage[spanish,es-tabla]{babel}
\selectlanguage{spanish}
\usepackage[utf8]{inputenc}
\usepackage[T1]{fontenc}
\usepackage{lmodern} % Scalable font
\usepackage{microtype}
\usepackage{placeins}

\RequirePackage{booktabs}
\RequirePackage[table]{xcolor}
\RequirePackage{xtab}
\RequirePackage{multirow}

% Links
\PassOptionsToPackage{hyphens}{url}\usepackage[colorlinks]{hyperref}
\hypersetup{
	allcolors = {red}
}

% Ecuaciones
\usepackage{amsmath}

% Rutas de fichero / paquete
\newcommand{\ruta}[1]{{\sffamily #1}}

% Párrafos
\nonzeroparskip

% Huérfanas y viudas
\widowpenalty100000
\clubpenalty100000

% Imágenes

% Comando para insertar una imagen en un lugar concreto.
% Los parámetros son:
% 1 --> Ruta absoluta/relativa de la figura
% 2 --> Texto a pie de figura
% 3 --> Tamaño en tanto por uno relativo al ancho de página
\usepackage{graphicx}
\newcommand{\imagen}[3]{
	\begin{figure}[!h]
		\centering
		\includegraphics[width=#3\textwidth]{#1}
		\caption{#2}\label{fig:#1}
	\end{figure}
	\FloatBarrier
}

% Comando para insertar una imagen sin posición.
% Los parámetros son:
% 1 --> Ruta absoluta/relativa de la figura
% 2 --> Texto a pie de figura
% 3 --> Tamaño en tanto por uno relativo al ancho de página
\newcommand{\imagenflotante}[3]{
	\begin{figure}
		\centering
		\includegraphics[width=#3\textwidth]{#1}
		\caption{#2}\label{fig:#1}
	\end{figure}
}

% El comando \figura nos permite insertar figuras comodamente, y utilizando
% siempre el mismo formato. Los parametros son:
% 1 --> Porcentaje del ancho de página que ocupará la figura (de 0 a 1)
% 2 --> Fichero de la imagen
% 3 --> Texto a pie de imagen
% 4 --> Etiqueta (label) para referencias
% 5 --> Opciones que queramos pasarle al \includegraphics
% 6 --> Opciones de posicionamiento a pasarle a \begin{figure}
\newcommand{\figuraConPosicion}[6]{%
  \setlength{\anchoFloat}{#1\textwidth}%
  \addtolength{\anchoFloat}{-4\fboxsep}%
  \setlength{\anchoFigura}{\anchoFloat}%
  \begin{figure}[#6]
    \begin{center}%
      \Ovalbox{%
        \begin{minipage}{\anchoFloat}%
          \begin{center}%
            \includegraphics[width=\anchoFigura,#5]{#2}%
            \caption{#3}%
            \label{#4}%
          \end{center}%
        \end{minipage}
      }%
    \end{center}%
  \end{figure}%
}

%
% Comando para incluir imágenes en formato apaisado (sin marco).
\newcommand{\figuraApaisadaSinMarco}[5]{%
  \begin{figure}%
    \begin{center}%
    \includegraphics[angle=90,height=#1\textheight,#5]{#2}%
    \caption{#3}%
    \label{#4}%
    \end{center}%
  \end{figure}%
}
% Para las tablas
\newcommand{\otoprule}{\midrule [\heavyrulewidth]}
%
% Nuevo comando para tablas pequeñas (menos de una página).
\newcommand{\tablaSmall}[5]{%
 \begin{table}
  \begin{center}
   \rowcolors {2}{gray!35}{}
   \begin{tabular}{#2}
    \toprule
    #4
    \otoprule
    #5
    \bottomrule
   \end{tabular}
   \caption{#1}
   \label{tabla:#3}
  \end{center}
 \end{table}
}

%
% Nuevo comando para tablas pequeñas (menos de una página).
\newcommand{\tablaSmallSinColores}[5]{%
 \begin{table}[H]
  \begin{center}
   \begin{tabular}{#2}
    \toprule
    #4
    \otoprule
    #5
    \bottomrule
   \end{tabular}
   \caption{#1}
   \label{tabla:#3}
  \end{center}
 \end{table}
}

\newcommand{\tablaApaisadaSmall}[5]{%
\begin{landscape}
  \begin{table}
   \begin{center}
    \rowcolors {2}{gray!35}{}
    \begin{tabular}{#2}
     \toprule
     #4
     \otoprule
     #5
     \bottomrule
    \end{tabular}
    \caption{#1}
    \label{tabla:#3}
   \end{center}
  \end{table}
\end{landscape}
}

%
% Nuevo comando para tablas grandes con cabecera y filas alternas coloreadas en gris.
\newcommand{\tabla}[6]{%
  \begin{center}
    \tablefirsthead{
      \toprule
      #5
      \otoprule
    }
    \tablehead{
      \multicolumn{#3}{l}{\small\sl continúa desde la página anterior}\\
      \toprule
      #5
      \otoprule
    }
    \tabletail{
      \hline
      \multicolumn{#3}{r}{\small\sl continúa en la página siguiente}\\
    }
    \tablelasttail{
      \hline
    }
    \bottomcaption{#1}
    \rowcolors {2}{gray!35}{}
    \begin{xtabular}{#2}
      #6
      \bottomrule
    \end{xtabular}
    \label{tabla:#4}
  \end{center}
}

%
% Nuevo comando para tablas grandes con cabecera.
\newcommand{\tablaSinColores}[6]{%
  \begin{center}
    \tablefirsthead{
      \toprule
      #5
      \otoprule
    }
    \tablehead{
      \multicolumn{#3}{l}{\small\sl continúa desde la página anterior}\\
      \toprule
      #5
      \otoprule
    }
    \tabletail{
      \hline
      \multicolumn{#3}{r}{\small\sl continúa en la página siguiente}\\
    }
    \tablelasttail{
      \hline
    }
    \bottomcaption{#1}
    \begin{xtabular}{#2}
      #6
      \bottomrule
    \end{xtabular}
    \label{tabla:#4}
  \end{center}
}

%
% Nuevo comando para tablas grandes sin cabecera.
\newcommand{\tablaSinCabecera}[5]{%
  \begin{center}
    \tablefirsthead{
      \toprule
    }
    \tablehead{
      \multicolumn{#3}{l}{\small\sl continúa desde la página anterior}\\
      \hline
    }
    \tabletail{
      \hline
      \multicolumn{#3}{r}{\small\sl continúa en la página siguiente}\\
    }
    \tablelasttail{
      \hline
    }
    \bottomcaption{#1}
  \begin{xtabular}{#2}
    #5
   \bottomrule
  \end{xtabular}
  \label{tabla:#4}
  \end{center}
}



\definecolor{cgoLight}{HTML}{EEEEEE}
\definecolor{cgoExtralight}{HTML}{FFFFFF}

%
% Nuevo comando para tablas grandes sin cabecera.
\newcommand{\tablaSinCabeceraConBandas}[5]{%
  \begin{center}
    \tablefirsthead{
      \toprule
    }
    \tablehead{
      \multicolumn{#3}{l}{\small\sl continúa desde la página anterior}\\
      \hline
    }
    \tabletail{
      \hline
      \multicolumn{#3}{r}{\small\sl continúa en la página siguiente}\\
    }
    \tablelasttail{
      \hline
    }
    \bottomcaption{#1}
    \rowcolors[]{1}{cgoExtralight}{cgoLight}

  \begin{xtabular}{#2}
    #5
   \bottomrule
  \end{xtabular}
  \label{tabla:#4}
  \end{center}
}



\graphicspath{ {./img/} }

% Capítulos
\chapterstyle{bianchi}
\newcommand{\capitulo}[2]{
	\setcounter{chapter}{#1}
	\setcounter{section}{0}
	\setcounter{figure}{0}
	\setcounter{table}{0}
	\chapter*{\thechapter.\enskip #2}
	\addcontentsline{toc}{chapter}{\thechapter.\enskip #2}
	\markboth{#2}{#2}
}

% Apéndices
\renewcommand{\appendixname}{Apéndice}
\renewcommand*\cftappendixname{\appendixname}

\newcommand{\apendice}[1]{
	%\renewcommand{\thechapter}{A}
	\chapter{#1}
}

\renewcommand*\cftappendixname{\appendixname\ }

% Formato de portada
\makeatletter
\usepackage{xcolor}
\newcommand{\tutor}[1]{\def\@tutor{#1}}
\newcommand{\course}[1]{\def\@course{#1}}
\definecolor{cpardoBox}{HTML}{E6E6FF}
\def\maketitle{
  \null
  \thispagestyle{empty}
  % Cabecera ----------------
\noindent\includegraphics[width=\textwidth]{cabecera}\vspace{1cm}%
  \vfill
  % Título proyecto y escudo informática ----------------
  \colorbox{cpardoBox}{%
    \begin{minipage}{.8\textwidth}
      \vspace{.5cm}\Large
      \begin{center}
      \textbf{TFG del Grado en Ingeniería Informática}\vspace{.6cm}\\
      \textbf{\LARGE Sistema de Procesamiento de Vídeo}
      \end{center}
      \vspace{.2cm}
    \end{minipage}

  }%
  \hfill\begin{minipage}{.20\textwidth}
    \includegraphics[width=\textwidth]{escudoInfor}
  \end{minipage}
  \vfill
  % Datos de alumno, curso y tutores ------------------
  \begin{center}%
  {%
    \noindent\LARGE
    Presentado por Álvaro Márquez\\ 
    en Universidad de Burgos --- \@date{}\\
    Tutores: D. José Miguel Ramírez Sanz y D. José Luís Garrido Labrador\\
  }%
  \end{center}%
  \null
  \cleardoublepage
  }
\makeatother

\newcommand{\nombre}{Nombre del alumno} %%% cambio de comando

% Datos de portada
\title{título del TFG}
\author{\nombre}
\tutor{nombre tutor}
\date{\today}

\begin{document}

\maketitle


\newpage\null\thispagestyle{empty}\newpage


%%%%%%%%%%%%%%%%%%%%%%%%%%%%%%%%%%%%%%%%%%%%%%%%%%%%%%%%%%%%%%%%%%%%%%%%%%%%%%%%%%%%%%%%
\thispagestyle{empty}


\noindent\includegraphics[width=\textwidth]{cabecera}\vspace{1cm}

\noindent D. José Miguel Ramírez Sanz y D. José Luís Garrido Labrador, profesores del Departamento de Ingeniería Informática, área de Lenguajes y Sistemas Informáticos.

\noindent Expone:

\noindent Que el alumno D. Álvaro Márquez, con DNI 71362377R, ha realizado el Trabajo final de Grado en Ingeniería Informática titulado Sistema de Procesamiento de Vídeo. 

\noindent Y que dicho trabajo ha sido realizado por el alumno bajo la dirección del que suscribe, en virtud de lo cual se autoriza su presentación y defensa.

\begin{center} %\large
En Burgos, {\large \today}
\end{center}

\vfill\vfill\vfill

% Author and supervisor
\begin{minipage}{0.45\textwidth}
\begin{flushleft} %\large
Vº. Bº. del Tutor:\\[2cm]
D. José Luís Garrido Labrador
\end{flushleft}
\end{minipage}
\hfill
\begin{minipage}{0.45\textwidth}
\begin{flushleft} %\large
Vº. Bº. del co-tutor:\\[2cm]
D. José Miguel Ramírez Sanz
\end{flushleft}
\end{minipage}
\hfill

\vfill

% para casos con solo un tutor comentar lo anterior
% y descomentar lo siguiente
%Vº. Bº. del Tutor:\\[2cm]
%D. nombre tutor


\newpage\null\thispagestyle{empty}\newpage




\frontmatter

% Abstract en castellano
\renewcommand*\abstractname{Resumen}
\begin{abstract}
En el contexto tecnológico actual, la generación de datos de vídeo ha crecido de forma exponencial, creando la necesidad de desarrollar sistemas robustos, escalables y eficientes para su procesamiento. Este proyecto aborda dicho desafío mediante el diseño y la implementación de una infraestructura de software completa y de extremo a extremo, desde la captura del vídeo hasta su análisis.

El sistema utiliza Jitsi Meet como plataforma de videoconferencia y su componente Jibri para la grabación de las sesiones, generando los ficheros de vídeo fuente. Para la ingesta y el procesamiento, se ha desplegado un pipeline de datos que utiliza Apache Kafka como broker de mensajería en su modo moderno KRaft y un clúster de Apache Spark para la computación distribuida. Toda la arquitectura, compuesta por más de siete servicios independientes, se ha desplegado y orquestado mediante Docker y Docker Compose, garantizando la reproducibilidad y portabilidad del entorno.

El proyecto culmina con una prueba de concepto funcional que demuestra la viabilidad del pipeline: un vídeo grabado con Jitsi es consumido y procesado por un trabajo de Spark escrito en Python, que le aplica una transformación visual (un tinte de color de pantalla dividida) utilizando la librería OpenCV, validando así el flujo de datos completo.
\end{abstract}

\renewcommand*\abstractname{Descriptores}
\begin{abstract}
Procesamiento de Vídeo, Big Data, Arquitectura de Microservicios, Apache Kafka, Apache Spark, Jitsi, Jibri, Docker, Docker Compose, Telerehabilitación, Visión Artificial, Pipeline de Datos.
\end{abstract}

\clearpage

% Abstract en inglés
\renewcommand*\abstractname{Abstract}
\begin{abstract}
In the current technological context, video data generation has grown exponentially, creating the need to develop robust, scalable, and efficient systems for its processing. This project addresses this challenge by designing and implementing a complete, end-to-end software infrastructure, from video capture to its analysis.

The system uses Jitsi Meet as a videoconferencing platform and its Jibri component for recording sessions, generating the source video files. For ingestion and processing, a data pipeline has been deployed using Apache Kafka as a messaging broker in its modern KRaft mode and an Apache Spark cluster for distributed computing. The entire architecture, comprising more than seven independent services, has been deployed and orchestrated using Docker and Docker Compose, ensuring the reproducibility and portability of the environment.

The project culminates in a functional proof of concept that demonstrates the pipeline's viability: a video recorded with Jitsi is consumed and processed by a Spark job written in Python, which applies a visual transformation (a split-screen color tint) using the OpenCV library, thus validating the entire data flow.
\end{abstract}

\renewcommand*\abstractname{Keywords}
\begin{abstract}
Video Processing, Big Data, Microservices Architecture, Apache Kafka, Apache Spark, Jitsi, Jibri, Docker, Docker Compose, Telerehabilitation, Computer Vision, Data Pipeline.
\end{abstract}

\clearpage

% Indices
\tableofcontents

\clearpage

\listoffigures

\clearpage

\listoftables
\clearpage

\mainmatter
\chapter{Introducción}
\label{chap:introduccion}

La enfermedad de Parkinson es un trastorno neurodegenerativo que afecta a millones de personas en todo el mundo, con una incidencia especialmente alta en personas de edad avanzada \cite{fuente_parkinson_epidemiologia}. Para estos pacientes, la rehabilitación física es clave para mantener su calidad de vida, pero el acceso a estas terapias a menudo supone un desafío considerable. Esta problemática se agrava en zonas rurales o despobladas, donde la distancia a los centros de salud y la dependencia de terceros para el desplazamiento complican o incluso impiden la continuidad del tratamiento, afectando negativamente a la evolución del paciente.

En este contexto, la telerehabilitación surge como una solución cada vez más necesaria para superar las dificultades de desplazamiento y la saturación de los servicios sanitarios \cite{cottrell2017telerehabilitation}. Este Trabajo de Fin de Grado se enmarca en el proyecto de investigación \textbf{FIS PI19/00670} \cite {garrido_fisfbis}, en colaboración con el Hospital Universitario de Burgos (HUBU), que busca precisamente mejorar las terapias a distancia para estos pacientes.

El punto de partida de este trabajo es un sistema previo, desarrollado en el marco del proyecto mencionado, que ya permitía un análisis básico de los ejercicios grabados en vídeo. Sin embargo, su infraestructura tecnológica presentaba limitaciones de escalabilidad y fiabilidad que dificultaban su uso a gran escala. El objetivo principal de este TFG es, por tanto, el rediseño y la construcción de un \textit{backend} mucho más robusto, capaz de procesar múltiples flujos de vídeo de forma eficiente y estable, sentando así las bases para un futuro sistema de \textit{feedback} automático fiable.

Para conseguirlo, se ha diseñado e implementado una nueva arquitectura basada en los paradigmas de microservicios y procesamiento de datos a gran escala. La solución implementa un \textit{pipeline} de datos que orquesta la captura de las sesiones de vídeo, su transporte a través de un sistema de mensajería distribuida y su posterior análisis mediante un motor de computación en clúster. Toda esta infraestructura se ha empaquetado mediante tecnologías de contenerización para garantizar un despliegue consistente, reproducible y fácilmente escalable.

La presente memoria documenta todo el proceso de investigación, diseño y desarrollo llevado a cabo. El Capítulo \ref{chap:objetivos} detalla los objetivos generales y técnicos del proyecto. En el Capítulo \ref{chap:conceptos} se exponen los fundamentos teóricos de las tecnologías empleadas. El Capítulo \ref{chap:herramientas} describe en profundidad las herramientas y la metodología de trabajo utilizadas. El Capítulo \ref{chap:desarrollo} narra los aspectos más relevantes del desarrollo, incluyendo los desafíos encontrados y las decisiones de diseño tomadas. Finalmente, el Capítulo \ref{chap:conclusiones} presenta las conclusiones extraídas del trabajo realizado y propone posibles líneas de trabajo futuras.
\capitulo{2}{Objetivos del proyecto}

En este capítulo se definen las metas y los objetivos que se han perseguido a lo largo del desarrollo de este Trabajo de Fin de Grado. Estos se han dividido en objetivos generales, que describen el propósito principal del proyecto, y objetivos técnicos, que detallan las metas específicas a nivel de implementación y tecnología. Finalmente, se incluye una sección con los objetivos personales que se han originado con la realización de este trabajo.

\section{Objetivos generales}

El objetivo principal de este proyecto es el diseño y la implementación de una infraestructura de \textit{backend}para un sistema de procesamiento de vídeo destinado a la tele-rehabilitación de pacientes con la enfermedad de Parkinson.

Este objetivo general se fundamenta en la necesidad de evolucionar un sistema preexistente, asegurando que la nueva arquitectura pueda manejar de forma fiable la captura, el transporte y el almacenamiento de los flujos de datos de vídeo, sentando así las bases para futuras implementaciones de algoritmos de análisis y \textit{feedback} automático.

\section{Objetivos técnicos}

Para alcanzar el objetivo general, se establecieron las siguientes metas técnicas específicas:

\begin{itemize}
    \item Investigar y seleccionar un conjunto de tecnologías de código abierto adecuadas para construir un \textit{pipeline} de datos en \textit{streaming}, centrándose en sus licencias.
    
    \item Desplegar y configurar un sistema de captura de vídeo basado en \textbf{Jitsi}, utilizando su componente \textbf{Jibri} para la grabación de las sesiones de ejercicios en formato de archivo (MP4).
    
    \item Utilizar \textbf{Apache Kafka} como un bus de mensajería distribuido para desacoplar los componentes del sistema y garantizar un transporte de datos fiable.
    
    \item Integrar \textbf{Apache Spark} en la arquitectura como motor de procesamiento, preparando el sistema para el futuro análisis de los datos de vídeo almacenados o en \textit{streaming}.
    
    \item Contenerizar toda la infraestructura utilizando \textbf{Docker} y \textbf{Docker Compose}, con el fin de crear un entorno de despliegue consistente, reproducible y fácil de gestionar.
    
    \item Documentar de manera exhaustiva todo el proceso de investigación, diseño, implementación y los desafíos técnicos encontrados, utilizando \LaTeX{} a través de la plataforma Overleaf.
    
    \item Utilizar herramientas modernas de gestión de proyectos (\textbf{Jira}) y control de versiones (\textbf{Git/GitHub}) para llevar un seguimiento riguroso del trabajo realizado y asegurar la integridad del código fuente y la documentación.
    
\end{itemize}


\section{Objetivos personales}

Más allá de los requisitos técnicos, la realización de este TFG permite perseguir una serie de objetivos personales y de desarrollo profesional:

\begin{itemize}
    \item Aplicar los conocimientos teóricos adquiridos a un problema de ingeniería complejo y real.
    \item Profundizar y ganar experiencia práctica en tecnologías de alta demanda en la industria, como son las plataformas(Kafka, Spark) y las herramientas (Docker).
    \item Afrontar y resolver problemas técnicos, desarrollando la capacidad de análisis, depuración y toma de decisiones ante imprevistos, como los surgidos durante el despliegue de la infraestructura.
    \item Contribuir, aunque sea a nivel de infraestructura, a un proyecto con un impacto social positivo y relevante.
\end{itemize}
\chapter{Conceptos Teóricos}
\label{chap:conceptos}

Para comprender las decisiones de diseño y la implementación de la infraestructura de este TFG, es necesario primero asentar las bases teóricas sobre las que se construyen las tecnologías seleccionadas. Este capítulo profundiza en los paradigmas y arquitecturas clave del procesamiento de datos distribuidos, el streaming de vídeo y los conceptos de red y seguridad que son el núcleo de este proyecto.

\section{Paradigmas de Arquitecturas Distribuidas}
\label{sec:conceptos_arquitecturas}
Un sistema distribuido se compone de múltiples componentes de software autónomos, ejecutándose en diferentes nodos, que se comunican y coordinan entre sí para cumplir un objetivo común. El diseño de este tipo de sistemas se basa en una serie de patrones y paradigmas que garantizan su eficiencia, escalabilidad y tolerancia a fallos.

\subsection{Sistemas de Mensajería: El Modelo Publicar-Suscribir}
El modelo de comunicación Publicar-Suscribir (o \textit{Pub-Sub}) es un patrón de mensajería asíncrono donde las aplicaciones que envían mensajes, llamadas \textbf{productores} (\textit{publishers}), no los envían directamente a los receptores. En su lugar, los publican en canales o categorías lógicas, conocidas como \textbf{tópicos} (\textit{topics}), gestionados por un intermediario o \textit{broker}. Por otro lado, las aplicaciones que reciben los mensajes, llamadas \textbf{consumidores} (\textit{subscribers}), se suscriben a los tópicos de su interés para recibir los mensajes que se publican en ellos \cite{birman2007guide}.

La principal ventaja de este patrón es el \textbf{desacoplamiento} que introduce entre los componentes. Los productores no necesitan conocer a los consumidores, y viceversa. Esto permite que los sistemas escalen de forma independiente y que nuevos componentes puedan añadirse a la arquitectura sin necesidad de modificar los existentes, dota a la arquitectura de una enorme flexibilidad y escalabilidad \cite{kleppmann2017designing}.

\subsection{Virtualización a Nivel de Sistema Operativo}
La virtualización es una técnica que permite crear versiones virtuales de recursos informáticos. Mientras que las máquinas virtuales tradicionales emulan un sistema de hardware completo sobre el que se ejecuta un sistema operativo invitado, la \textbf{virtualización a nivel de sistema operativo}, más conocida como contenerización, sigue un enfoque diferente.

Los contenedores se ejecutan sobre el kernel del sistema operativo anfitrión, pero en espacios de usuario aislados. Cada contenedor tiene su propio sistema de ficheros, procesos y configuración de red, pero comparte el mismo kernel subyacente. Esto los hace extremadamente ligeros, rápidos de iniciar y mucho más eficientes en el uso de recursos que las máquinas virtuales, convirtiéndolos en la tecnología ideal para desplegar arquitecturas de microservicios.

\section{Fundamentos de Tecnologías de Streaming}
\label{sec:conceptos_streaming}

\subsection{El Log de Transacciones Distribuido e Inmutable}
Una de las arquitecturas más potentes para construir sistemas de mensajería y plataformas de \textit{streaming} de datos es el \textbf{log de transacciones distribuido} (o \textit{commit log}). Conceptualmente, es una estructura de datos inmutable a la que solo se pueden añadir registros. Una vez escrito, un registro no puede ser modificado ni eliminado.

En un sistema distribuido, este log se divide en \textbf{particiones} para permitir el paralelismo y la escalabilidad. Cada partición es un log ordenado que se replica a través de múltiples servidores (\textbf{brokers}) para garantizar la tolerancia a fallos y la alta disponibilidad. Cada registro en una partición tiene un identificador secuencial único llamado \textbf{offset}, que permite a los consumidores llevar un control preciso de su posición de lectura, incluso en caso de fallos. Este modelo es la base de sistemas de mensajería de alto rendimiento capaces de procesar millones de eventos por segundo.

\subsection{Modelo de Computación Distribuida en Memoria}
El procesamiento de grandes volúmenes de datos (\textit{Big Data}) requiere modelos de computación que puedan paralelizar el trabajo a través de un clúster de máquinas. La abstracción fundamental en los motores de procesamiento modernos es el \textbf{Conjunto de Datos Distribuido y Resiliente} (\textit{Resilient Distributed Dataset} o RDD). Un RDD es una colección de elementos inmutable y tolerante a fallos que puede ser procesada en paralelo \cite{zaharia2012resilient}.

Las operaciones sobre estos datos se dividen en:
\begin{itemize}
    \item \textbf{Transformaciones:} Operaciones que crean un nuevo conjunto de datos a partir de uno existente (ej. un filtrado). Son "perezosas" (\textit{lazy}), es decir, su cómputo no se realiza al momento.
    \item \textbf{Acciones:} Operaciones que disparan la ejecución de las transformaciones y devuelven un resultado o escriben en un sistema externo.
\end{itemize}
Cuando se invoca una acción, el motor de procesamiento analiza la cadena de transformaciones y construye un \textbf{Grafo Acíclico Dirigido (DAG)} de las operaciones. Este grafo es optimizado y dividido en etapas de tareas que se distribuyen entre los nodos del clúster para su ejecución paralela, permitiendo un procesamiento masivo y eficiente.

\subsection{Transmisión de Vídeo sobre IP}
La transmisión de vídeo en tiempo real a través de redes IP se basa en protocolos diseñados para manejar datos multimedia con baja latencia. Uno de los protocolos más extendidos para este fin es el \textbf{Protocolo de Mensajería en Tiempo Real (RTMP)}. Desarrollado originalmente para Adobe Flash, RTMP es un protocolo basado en TCP que mantiene una sesión persistente entre cliente y servidor, permitiendo la transmisión fiable de flujos de audio, vídeo y datos.

\section{Conceptos de Seguridad en Comunicaciones Web}
\subsection{HTTP, HTTPS y el Protocolo SSL/TLS}
HTTP (Hypertext Transfer Protocol) es el protocolo fundamental de la web, pero transmite la información en texto plano. \textbf{HTTPS} es su versión segura, que añade una capa de cifrado mediante el protocolo \textbf{SSL/TLS} (Secure Sockets Layer/Transport Layer Security) \cite{rescorla2001ssl}. Esta capa utiliza criptografía de clave pública para establecer una conexión segura, garantizando la confidencialidad (nadie puede leer la comunicación) y la integridad (nadie puede modificarla) de los datos.

\subsection{Certificados Digitales y Autoridades de Certificación}
Para que HTTPS funcione, el servidor debe presentar un \textbf{certificado digital SSL/TLS} que valide su identidad. Existen dos tipos principales:
\begin{itemize}
    \item \textbf{Certificados Autofirmados:} Son generados por el propio administrador del servidor. No son emitidos por una entidad de confianza, por lo que los navegadores web los rechazan por defecto, mostrando advertencias de seguridad al usuario.
    \item \textbf{Certificados de una CA:} Son emitidos por una \textbf{Autoridad de Certificación (CA)}, una entidad externa en la que confían los navegadores (como Let's Encrypt). Cuando un navegador recibe un certificado emitido por una CA de confianza, establece la conexión segura sin advertencias. Por motivos de seguridad, los navegadores modernos exigen una conexión HTTPS con un certificado válido para permitir el acceso a funcionalidades sensibles como la cámara o el micrófono.
\end{itemize}

\section{Componentes Conceptuales de un Pipeline de Datos}
\label{sec:conceptos_pipeline}

Toda arquitectura de procesamiento de datos, como la que se aborda en este proyecto, puede descomponerse en una serie de componentes lógicos, cada uno con una responsabilidad bien definida. La comprensión de estos roles es fundamental para diseñar un sistema robusto y escalable.

\subsection{El Productor de Eventos (\textit{Producer})}
El \textbf{Productor} es cualquier entidad del sistema cuya función principal es originar o capturar datos y enviarlos a un sistema de mensajería. Este componente actúa como la fuente de información del pipeline. En el contexto de un sistema de análisis de vídeo, el rol del productor sería desempeñado por el sistema de captura, encargado de digitalizar la sesión y publicarla como un evento o una serie de eventos en el flujo de datos.

\subsection{El Intermediario de Mensajería (\textit{Broker})}
El \textbf{Broker} es el servidor o conjunto de servidores que forman el núcleo del sistema de mensajería. Su responsabilidad es recibir los eventos de los productores, almacenarlos de forma duradera y fiable, y ponerlos a disposición de los consumidores. En arquitecturas distribuidas, se despliega un clúster de brokers para garantizar la alta disponibilidad y la tolerancia a fallos. Este componente es esencial para desacoplar a los productores de los consumidores, permitiendo que operen a ritmos diferentes y de forma independiente.

\subsection{El Consumidor y Procesador de Datos (\textit{Consumer}/\textit{Processor})}
El \textbf{Consumidor} es la aplicación que se suscribe al sistema de mensajería para recibir y leer los eventos que le interesan. Una vez que un consumidor recibe un evento, se lo entrega a una lógica de \textbf{Procesamiento}, que es donde reside la inteligencia de la aplicación. Esta lógica es la que se encarga de transformar, analizar o actuar sobre los datos recibidos. En un sistema de procesamiento de vídeo, el consumidor se encargaría de recibir los datos del vídeo, y el procesador aplicaría los algoritmos de visión artificial sobre ellos.

\subsection{El Modelo de Computación Maestro-Trabajador (\textit{Master-Worker})}
Para procesar grandes volúmenes de datos de manera eficiente, los \textit{frameworks} de computación distribuida suelen emplear el patrón arquitectónico \textbf{Maestro-Trabajador} (conocido también como \textit{Driver-Executor}).
\begin{itemize}
    \item El nodo \textbf{Maestro} (\textit{Master} o \textit{Driver}) es el cerebro de la operación. Recibe el trabajo a realizar, lo divide en un conjunto de tareas más pequeñas e independientes, y las distribuye entre los nodos trabajadores. También se encarga de coordinar la ejecución y agregar los resultados finales.
    \item Los nodos \textbf{Trabajadores} (\textit{Workers} o \textit{Executors}) son los que realizan el trabajo pesado. Cada trabajador recibe una o más tareas del maestro, las ejecuta en paralelo con otros trabajadores sobre una porción de los datos, y devuelve el resultado al maestro.
\end{itemize}
\capitulo{4}{Técnicas y herramientas}

\subsection{Metodología de Desarrollo}

Para organizar y gestionar el trabajo de este TFG de una manera flexible y adaptativa, se ha decidido emplear un enfoque ágil.

Las ideas principales de esta forma de trabajo aplicada al TFG son:

\begin{itemize}
    \item \textbf{Organización en Sprints:} El trabajo total se divide en periodos de tiempo, los Sprints. En cada Sprint, el objetivo es completar un conjunto concreto de tareas previamente seleccionadas.
    
    \item \textbf{Uso de Jira para la Gestión:} Se utiliza la herramienta de software \textbf{Jira} como soporte principal para la gestión del proyecto. En Jira creo:
        \begin{itemize}
            \item Una lista general de tareas pendientes (Backlog).
            \item La planificación de qué tareas se abordarán en cada Sprint.
            \item Un tablero visual para seguir el estado de las tareas del Sprint actual (si están Por Hacer, En Curso o ya Hechas).
            \item Se usan también Épicas para agrupar fases grandes del proyecto (como la investigación inicial, el desarrollo de componentes, la escritura de la documentación, etc.).
        \end{itemize}
    
    \item \textbf{Seguimiento y Adaptación:} Se mantienen reuniones periódicas (normalmente semanales) con los tutores para revisar el trabajo realizado en el Sprint, comentar los avances, resolver dudas o problemas...
\end{itemize}

Este enfoque ágil resulta beneficioso, ya que permite adaptarse mejor a los resultados de la investigación o a los desafíos técnicos que puedan surgir, a la vez que facilita un seguimiento continuo del progreso hacia los objetivos finales.

\subsubsection{Apache Kafka}

Apache Kafka es una plataforma distribuida de código abierto, ampliamente reconocida en la industria, diseñada específicamente para la gestión y procesamiento de flujos de eventos en tiempo real. Funciona como un sistema de mensajería publicar-suscribir de alto rendimiento, que garantiza durabilidad y tolerancia a fallos.

% Rol en este TFG
En el marco de este proyecto, Apache Kafka se perfila como una pieza central de la arquitectura. Su rol principal será actuar como el sistema intermediario encargado de recibir de forma eficiente los flujos de datos generados por la fuente de vídeo (Jitsi/Jibri) y almacenarlos temporalmente de manera fiable para que sean consumidos posteriormente por el motor de procesamiento (Apache Spark). Esto aportando flexibilidad y robustez al conjunto.

% Justificación / Por qué Kafka
La elección de Kafka se fundamenta en:
\begin{itemize}
    \item \textbf{Manejo de Alto Volumen y Baja Latencia:} Kafka está optimizado para ingestar y servir grandes cantidades de datos (como los que puede generar el vídeo) con un retardo mínimo.
    \item \textbf{Escalabilidad Nativa:} El sistema permite escalar la capacidad de Kafka para adaptarse a la carga de trabajo.
    \item \textbf{Fiabilidad:} Su naturaleza distribuida y la duplicación de datos proporcionan tolerancia a fallos.
\end{itemize}

% Zookeeper / KRaft
Es relevante mencionar que Kafka requiere de Apache Zookeeper para tareas de coordinación y gestión de metadatos. Aunque esta dependencia está siendo eliminada en las versiones más recientes mediante el protocolo KRaft (que simplifica la infraestructura), es un factor a tener en cuenta según la versión que finalmente se despliegue en el proyecto.

% Consideraciones y Alternativas
Una consideración técnica es que Kafka no está optimizado para manejar archivos de vídeo extremadamente grandes como mensajes únicos; se deben considerar estrategias como el envío de metadatos. Kafka es una opción preferente para este proyecto.
\chapter{Aspectos relevantes del desarrollo del proyecto}
\label{chap:desarrollo}

En este capítulo expongo la crónica del desarrollo de este TFG. Lejos de ser un proceso lineal, el trabajo ha seguido un camino iterativo, marcado por la investigación, la implementación de prototipos, la aparición de desafíos técnicos y la toma de decisiones estratégicas para superar los bloqueos. Esta narrativa busca reflejar fielmente el proceso de ingeniería real que he llevado a cabo, detallando no solo los éxitos, sino también los problemas y el proceso de depuración que ha sido necesario para resolverlos.

\section{Fase Inicial: Diseño de la Arquitectura y Selección de Tecnologías}
El proyecto comenzó con una fase de investigación y planificación, documentada en el Anexo \ref{apendice:plan_proyecto}. Durante esta etapa, se tomaron decisiones fundamentales sobre la pila tecnológica y el alcance inicial del proyecto.

\subsection{Selección de la Pila Tecnológica}
Basándome en los requisitos de construir un sistema de procesamiento de vídeo escalable, realicé un estudio de las tecnologías de Big Data más adecuadas, como se detalla en el Capítulo \ref{chap:tecnicas_herramientas}. La pila tecnológica seleccionada fue:
\begin{itemize}
    \item \textbf{Jitsi/Jibri:} Como solución de código abierto para la captura de las sesiones de vídeo. Opté por utilizar el proyecto oficial \texttt{docker-jitsi-meet}, ya que proporciona una configuración pre-empaquetada con Docker Compose que, teóricamente, simplifica el despliegue de todos sus micro servicios (web, prosody, jicofo, jvb, jibri).
    \item \textbf{Apache Kafka:} Como bus de mensajería para desacoplar la captura del procesamiento.
    \item \textbf{Apache Spark:} Como motor de procesamiento distribuido para el futuro análisis de los datos.
\end{itemize}

\subsection{Definición del Alcance: Enfoque Offline como Primer Hito}
Desde el principio, era consciente de la complejidad de un sistema de procesamiento en tiempo real. La integración de un flujo RTMP de Jibri con Kafka y Spark Streaming presenta desafíos técnicos significativos. Por ello, tomé una decisión estratégica clave para mitigar riesgos: \textbf{priorizar un flujo de trabajo offline como primer objetivo funcional}. En este enfoque, Jibri grabaría las sesiones en archivos MP4, que se almacenarían en un sistema de ficheros accesible para que Spark los procesara por lotes. Esto me permitió dividir el problema en fases manejables, asegurando la entrega de una prueba de concepto funcional antes de abordar la complejidad del streaming en tiempo real.

\section{Acto I: El Desafío del Entorno Windows con WSL2}
\label{sec:desarrollo_acto1}
El primer intento de despliegue se realizó sobre mi máquina de desarrollo principal: un sistema Windows 11 con el Subsistema de Windows para Linux (WSL2) y Docker Desktop. Aunque este entorno es muy versátil, la interacción entre el sistema de archivos de Windows (NTFS) y los permisos de Linux dentro de los contenedores demostró ser la fuente de problemas complejos y bloqueantes.

\subsection{Primer Despliegue de Jitsi y Análisis de Errores Críticos}
Tras clonar el repositorio de \texttt{docker-jitsi-meet} y configurar el fichero \texttt{.env}, el primer intento de levantar la pila de servicios con \texttt{docker-compose up} resultó en un fallo en cascada. Dos componentes clave, Prosody y Jibri, fallaban sistemáticamente.

\subsubsection{Errores de Permisos en el Contenedor Prosody}
El análisis de los logs del contenedor de Prosody (el servidor XMPP) fue el primer indicio claro del problema subyacente. Los registros mostraban de forma repetida y consistente errores de tipo \texttt{Permission denied}, como se puede observar en este extracto:
\begin{verbatim}[frame=single, label={Fragmento de log de error de Prosody}]
datamanager error Unable to write to accounts storage 
('/config/data/auth%2elocalhost/accounts/focus.dat~: Permission denied')
\end{verbatim}
Este error se debe a un conflicto en la gestión de permisos de ficheros entre sistemas operativos. El proceso de Prosody se ejecuta dentro del contenedor con un usuario y grupo específicos de Linux (\textit{uid/gid}), pero el volumen donde intenta escribir (\texttt{/config/data}) está montado desde el sistema de archivos NTFS de Windows a través de WSL2. La capa de traducción de permisos de WSL2 no era capaz de mapear correctamente los permisos, impidiendo que el proceso tuviera los privilegios de escritura necesarios. A pesar de intentar modificar los permisos en Windows y probar diferentes configuraciones de montaje de volúmenes en Docker, el problema persistió, indicando que se trataba de una limitación fundamental del entorno.

\subsubsection{Fallos de Autenticación en Cascada: Jibri y Jicofo}
El fallo de Prosody provocó un efecto dominó. \textbf{Jicofo}, el componente que gestiona las conferencias, no podía establecer conexión con el servidor XMPP. De forma aún más crítica, \textbf{Jibri}, el grabador de vídeo, fallaba con dos errores distintos:
\begin{itemize}
    \item Un error fatal \texttt{FATAL ERROR: Jibri recorder password and auth password must be set}, sugiriendo un problema en cómo se leían las variables de entorno desde el fichero \texttt{.env} en el entorno WSL2.
    \item Un error de autenticación \texttt{SASLError using SCRAM-SHA-1: malformed-request}, consecuencia directa de no poder conectar con un servidor Prosody que no había logrado arrancar correctamente.
\end{itemize}

\subsection{Decisión Estratégica: Pivote a un Entorno Linux Nativo}
Tras varios días de depuración, la conclusión fue clara: los problemas no eran de configuración de Jitsi, sino de la propia infraestructura de desarrollo basada en Windows y WSL2. Intentar resolver estas incompatibilidades de bajo nivel representaba un riesgo muy alto y una posible pérdida de tiempo que se desviaba de los objetivos del TFG.

Por ello, tomé la decisión de ingeniería más importante del proyecto hasta la fecha: \textbf{realizar un pivote estratégico y migrar todo el entorno a un sistema operativo Linux nativo}. Se eligió Debian 12 por su estabilidad y su amplio uso en servidores, lo que me proporcionaría un entorno predecible y libre de las capas de compatibilidad que estaban causando los problemas.

\section{Acto II: Implementación y Depuración Sistemática en Debian}
\label{sec:desarrollo_acto2}
La migración a un entorno Linux nativo fue un punto de inflexión en el proyecto. Como esperaba, los problemas de permisos relacionados con WSL2 desaparecieron de inmediato. Al ejecutar el comando \texttt{docker-compose up -d} en el directorio de \texttt{docker-jitsi-meet}, los cinco contenedores principales (prosody, jicofo, jvb, web y jibri) lograron arrancar y mantenerse en ejecución, un hito que no había sido posible alcanzar anteriormente.

Sin embargo, el éxito inicial solo reveló la siguiente capa de desafíos. A pesar de que los servicios estaban activos, la interfaz web de Jitsi no era accesible, mostrando un error de "La conexión ha fallado". Esto dio comienzo a un nuevo e intensivo ciclo de depuración, esta vez centrado en la configuración de la red y los componentes internos de Jitsi.

% S:captura 
\subsection{Proceso de Depuración de la Conexión HTTP}
Para solucionar el problema de conexión, apliqué un proceso de diagnóstico sistemático, analizando los logs de cada contenedor y el comportamiento de la aplicación web para aislar la causa raíz.

\subsubsection{Diagnóstico del Protocolo de Conexión (WSS)}
El primer hallazgo importante lo obtuve al inspeccionar la consola de desarrollador del navegador. Descubrí que, a pesar de que mi instancia estaba configurada para HTTP, el cliente web intentaba establecer una conexión segura mediante el protocolo WebSocket Secure (WSS). Esto fallaba porque el servidor no tenía configurado ningún certificado SSL/TLS. La solución pasaba por forzar al cliente a utilizar una conexión no segura.

\subsubsection{Solución y Error de Montaje: \texttt{custom.config.js}}
La documentación de Jitsi indica que se puede sobreescribir la configuración del cliente web mediante un archivo \texttt{custom.config.js}. Creé este archivo con el objetivo de deshabilitar la conexión segura. Sin embargo, al volver a lanzar los contenedores, el servicio web falló con un nuevo error: \texttt{...not a directory}. El problema era que Docker, al intentar montar mi archivo de configuración en un volumen, se encontraba con que el directorio de destino ya había sido creado por el script de inicio del contenedor. La solución fue asegurarme de crear la estructura de directorios y el fichero \texttt{custom.config.js} en la carpeta de configuración del host (\texttt{\textasciitilde{}/.jitsi-meet-cfg/web/}) \textit{antes} de ejecutar \texttt{docker-compose up} por primera vez.

\subsubsection{Autenticación de Usuarios Anónimos (Invitados)}
Una vez solucionado el problema anterior, logré acceder a la interfaz web, pero me enfrenté a un nuevo bloqueo: no podía crear una sala como usuario anónimo o invitado. Los logs de Prosody indicaron que la autenticación para el dominio de invitados no estaba configurada. Para solucionarlo, tuve que editar manualmente el fichero de configuración principal de Prosody (\texttt{prosody.cfg.lua}) y añadir un nuevo \texttt{VirtualHost} para \texttt{"guest.jitsi.localhost"}, habilitando explícitamente la autenticación anónima. Tras este cambio, finalmente logré crear una conferencia y establecer una comunicación de vídeo y audio funcional.

\subsection{Iteración hacia HTTPS y Estrategia Actual}
Con el sistema funcionando en HTTP, y siguiendo las recomendaciones de los tutores, el siguiente paso fue intentar una implementación con HTTPS.
\begin{itemize}
    \item \textbf{Intento con Certificados Autofirmados:} Mi primer enfoque fue utilizar certificados autofirmados generados con la herramienta \texttt{mkcert}. Sin embargo, esto me devolvió a un problema similar al de WSL2: el contenedor web de Jitsi (Nginx) no era capaz de leer o utilizar correctamente los certificados montados, provocando errores de SSL en el navegador.
    
    \item \textbf{Diagnóstico con OpenSSL:} Para diagnosticar el problema de forma precisa, utilicé la herramienta de línea de comandos \texttt{openssl s\_client} para conectarme directamente al puerto 443 del servidor. El análisis confirmó que el servidor no estaba presentando el certificado que yo le había proporcionado, sino uno por defecto, lo que indicaba un fallo en la configuración de Nginx o en cómo Docker montaba los certificados.
    
    \item \textbf{Estrategia Definitiva (Let's Encrypt):} Basado en esta experiencia y en las indicaciones de los tutores, la estrategia actual y definitiva del proyecto es implementar HTTPS utilizando certificados válidos emitidos por una autoridad de certificación reconocida como \textbf{Let's Encrypt}. Este enfoque, aunque requiere un dominio público, garantiza una solución estándar, robusta y confiable, eliminando todos los problemas de confianza del navegador. Las tareas técnicas asociadas a esta implementación forman parte de la fase final del proyecto.
\end{itemize}

%  documentación futura
Este capítulo se completará, una vez finalizada la fase de implementación técnica, con una sección dedicada al desarrollo e integración del pipeline con Apache Spark, siguiendo la misma estructura.

%  falta anadir FIGURAS, CAPS, DIAGRAMAS...
\chapter{Trabajos relacionados}
\label{chap:trabajos_relacionados}

Para contextualizar adecuadamente el alcance y la contribución de este TFG, es fundamental analizar los proyectos e investigaciones previas que han servido como base o que abordan problemas similares. Este capítulo revisa los trabajos más influyentes, desde el proyecto directo del que este TFG es una continuación, hasta investigaciones académicas que validan el uso de las tecnologías seleccionadas.

\section{Proyecto de Origen: El Sistema FIS-FBIS}
\label{sec:trab_rel_fisfbis}
Este Trabajo de Fin de Grado es una continuación directa del proyecto \textbf{FIS-FBIS} \cite{garrido_fisfbis}, un sistema de bajo coste para la telerehabilitación de pacientes con Parkinson desarrollado en la Universidad de Burgos. El objetivo de FIS-FBIS era crear una plataforma completa que permitiera la comunicación entre paciente y terapeuta, la grabación de sesiones y, fundamentalmente, la evaluación automática de los ejercicios.

El sistema original fue el resultado de dos Trabajos de Fin de Máster que funcionaban de manera conjunta y complementaria: uno enfocado en la infraestructura de datos (TFM-FIS-IF) y otro en la inteligencia artificial para el análisis de vídeo (TFM-FIS-IA). Aquí proporciono el articulo científico relacionado \cite{garrido2023fishub}.

\subsection{TFM-FIS-IF: Infraestructura de Datos para Procesamiento de Vídeo}
El trabajo de José Luis Garrido Labrador \cite{garrido_tfm_if} se centró en la creación de una arquitectura Big Data basada en colas de mensajería para el procesado de vídeo en tiempo real. Su sistema ya utilizaba \textbf{Apache Kafka} como bus de eventos y \textbf{Apache Spark} para el procesamiento. La ingesta de datos se realizaba mediante scripts de Python (`emitter.py`, `producer.py`) que simulaban un flujo de vídeo y lo enviaban a Kafka para ser procesado por un consumidor de Spark (`consumer.py`).

Si bien este TFM sentó las bases de la arquitectura de este proyecto, se trataba de una prueba de concepto.Este TFG toma esta arquitectura como punto de partida y la evoluciona, reemplazando los scripts de simulación por una integración real con \textbf{Jitsi/Jibri} y profesionalizando todo el despliegue mediante \textbf{Docker y Docker Compose} para garantizar un entorno robusto, escalable y fácil de gestionar.

\subsection{TFM-FIS-IA: Visión Artificial para Comparación de Posiciones}
De forma paralela, el trabajo de José Miguel Ramírez Sanz \cite{ramirez_tfm_ia} abordó la parte de inteligencia artificial del sistema. Su TFM se centró en el estudio del estado del arte de las herramientas de visión artificial para Python y en la implementación de un sistema de comparación de posiciones. El objetivo era extraer el esqueleto humano de los fotogramas de vídeo para poder comparar los ejercicios realizados por los pacientes con los de los terapeutas.

La sinergia entre ambos TFMs es clara: el TFM de infraestructura (IF) proporcionaba el pipeline de datos, mientras que el de inteligencia artificial (IA) aportaba la lógica de análisis. Este trabajo se centra en mejorar y modernizar la parte de infraestructura (IF) para que pueda dar un soporte más sólido a futuras aplicaciones de análisis como la desarrollada en el TFM-IA.

\subsection{Arquitecturas para el Análisis de Vídeo en Tiempo Real con Spark}
\label{subsec:trab_rel_academico}
Más allá de los proyectos del entorno de la UBU, se ha revisado la literatura académica para validar la elección de la arquitectura. Un ejemplo es el artículo \textit{Distributed Real-Time Video Stream Analytics on top of Spar}k \cite{karimov2018distributed}. En este trabajo, los autores proponen una arquitectura genérica para el análisis de flujos de vídeo en tiempo real utilizando Spark Streaming.

El estudio demuestra cómo Spark es capaz de procesar flujos de datos para realizar tareas como la detección de objetos. Su enfoque valida la elección de Spark como motor de procesamiento para este TFG, dada su capacidad para manejar la carga computacional del análisis de vídeo de forma distribuida. Sin embargo, este proyecto se diferencia y amplía este enfoque al:
\begin{itemize}
    \item Integrar una solución completa de captura de vídeo para telemedicina como es \textbf{Jitsi/Jibri}, en lugar de fuentes de vídeo genéricas.
    \item Implementar \textbf{Apache Kafka} como un bus de mensajería intermedio, lo que proporciona un mayor desacoplamiento y tolerancia a fallos entre la captura y el procesamiento.
    \item Enfocar la arquitectura en un caso de uso específico (la telerehabilitación), en lugar de en el análisis genérico de vídeo.
\end{itemize}
\chapter{Conclusiones y Líneas de trabajo futuras}
\label{chap:conclusiones}

Para finalizar esta memoria, en este capítulo se presentan, por una parte, las conclusiones principales que se han podido extraer del desarrollo del proyecto y, por otra, las posibles líneas de trabajo futuras que se abren a partir de la infraestructura implementada.

\section{Conclusiones}

La realización de este TFG ha permitido obtener una serie de conclusiones tanto a nivel técnico como a nivel de gestión y desarrollo de un proyecto de ingeniería de software.

Como primera conclusión, y una de las más importantes, se ha podido constatar que la \textbf{elección del entorno de desarrollo es crítica} en proyectos que involucran múltiples servicios de red y sistemas de ficheros, como es este caso. El intento inicial de despliegue sobre Windows con WSL2, aunque teóricamente viable, demostró ser una fuente de problemas de compatibilidad (especialmente de permisos y redes) muy difíciles de diagnosticar y resolver. El pivote estratégico a un entorno Linux nativo (Debian 12) fue una decisión fundamental que no solo solucionó los problemas, sino que validó la importancia de trabajar sobre un sistema operativo estándar en entornos de servidor para garantizar la estabilidad y la reproducibilidad.

En segundo lugar, se ha logrado el objetivo principal de diseñar e implementar una \textbf{arquitectura de datos robusta y escalable}, sentando las bases para un sistema de telerehabilitación mucho más potente que el del proyecto original. La integración de Jitsi, Docker, Kafka y Spark conforma un pipeline de datos modular y desacoplado, que es el estándar de facto en la industria para este tipo de aplicaciones. Aunque la implementación final se ha centrado en un flujo \textit{offline}, la arquitectura está preparada para evolucionar hacia el procesamiento en tiempo real.

Finalmente, una gran parte del trabajo de ingeniería de este TFG no ha sido la programación de algoritmos complejos, sino la \textbf{integración, configuración y depuración} de múltiples tecnologías de código abierto. Enfrentarme a \textit{logs} de errores, diagnosticar problemas de red con herramientas como \texttt{openssl}, y comprender las complejas interacciones entre los distintos contenedores ha sido un aprendizaje práctico inmenso y una parte tan valiosa como el propio desarrollo de código.

\subsection{Conclusiones Personales}
A nivel personal, este Trabajo de Fin de Grado ha supuesto un desafío considerable y, a su vez, una experiencia de aprendizaje inmensamente valiosa. Me ha permitido aplicar los conocimientos teóricos del Grado y los obtenidos por otro medios a un problema real y complejo, enfrentándome a las dificultades que surgen fuera del entorno académico. La necesidad de investigar, diagnosticar errores y tomar decisiones estratégicas para reconducir el proyecto me ha proporcionado una visión práctica de la ingeniería de software que considero fundamental para mi futuro profesional.

\section{Líneas de trabajo futuras}

El sistema implementado en este TFG es una base sólida sobre la que se pueden construir numerosas mejoras y nuevas funcionalidades. A continuación, se detallan algunas de las líneas de trabajo futuras más interesantes:

\begin{enumerate}
    \item \textbf{Implementar el \textit{Pipeline} de \textit{Streaming} en Tiempo Real:} El siguiente paso natural sería evolucionar del procesamiento por lotes (\textit{batch}) al procesamiento en tiempo real. Esto implicaría configurar Jibri para que emita un flujo de vídeo usando el protocolo RTMP, e implementar un consumidor en Spark \textit{Streaming} que procese los fotogramas a medida que llegan a través de Kafka.
    
    \item \textbf{Integración con los Algoritmos de Análisis de Movimiento:} Conectar este \textit{pipeline} de datos con los algoritmos de análisis de esqueletos y DTW desarrollados en los trabajos previos (TFM-FIS-IA y el TFG de Lucía Núñez Calvo). Esto permitiría crear un sistema completo que capture, procese y evalúe los ejercicios de forma totalmente automatizada.
    
    \item \textbf{Escalabilidad y Orquestación Avanzada:} Aunque Docker Compose es ideal para el desarrollo, para un entorno de producción se podría migrar la arquitectura a un orquestador de contenedores más avanzado como \textbf{Kubernetes}. Esto permitiría un escalado automático de los componentes (por ejemplo, añadir más workers de Spark o más instancias de Jibri si hay muchas sesiones concurrentes).
    
    \item \textbf{Monitorización del Sistema:} Desplegar una pila de monitorización (por ejemplo, con Prometheus y Grafana) para obtener métricas en tiempo real del estado de los brokers de Kafka, los trabajos de Spark y el resto de los servicios, asegurando la salud y el rendimiento del sistema.

    \item \textbf{Mejora de la Captura de Vídeo y Audio:} Investigar y optimizar la calidad de la fuente de datos. Esto podría incluir pruebas con diferentes configuraciones de \textit{códecs} en Jibri, el uso de cámaras de mayor resolución, o la exploración de configuraciones de audio avanzadas para garantizar que la materia prima para el análisis sea de la máxima calidad posible.
    
\end{enumerate}



\bibliographystyle{plain}
\bibliography{bibliografia}

\end{document}
